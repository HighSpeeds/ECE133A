\documentclass[12pt]{article}
\author{Lawrence Liu}
\usepackage{subcaption}
\usepackage{graphicx}
\usepackage{amsmath}
\usepackage{pdfpages}
\newcommand{\Laplace}{\mathscr{L}}
\setlength{\parskip}{\baselineskip}%
\setlength{\parindent}{0pt}%
\usepackage{xcolor}
\usepackage{listings}
%\definecolor{backcolour}{rgb}{0.95,0.95,0.92}
\usepackage{amssymb}
\usepackage[T1]{fontenc}
\usepackage{beramono}%\lstdefinestyle{mystyle}{
%    backgroundcolor=\color{backcolour}}
%\lstset{style=mystyle}
%\usepackage[usenames,dvipsnames]{xcolor}%%
%% Julia definition (c) 2014 Jubobs
%%



\title{ECE 133A HW 6}
\begin{document}
\maketitle
\section*{Exercise A11.8}
\subsection*{(c)}
We have that 
$$R_11=1$$
And thus 
$$R_{1,2:3}=[0,1]$$
Thus now we need to compute the cholensky factorization of 
$$\begin{bmatrix}
    1 & 1\\
    1 & a
\end{bmatrix}-\begin{bmatrix}
    0 & 1\\
    1 & 0
\end{bmatrix}=\begin{bmatrix}
    1 & 0\\
    0 & a
\end{bmatrix}$$
Thus we have that 
$$R_{22}=1$$
And thus we have that 
$$R_{23}=0$$
And thus we have that
$$R_{33}=\sqrt{a}$$
Thus we have that A is positive definite if and only if $a\geq0$, and if it 
exists we have
$$R=\begin{bmatrix}
    1 & 0 & 1\\
    0 & 1 & 0\\
    0 & 0 & \sqrt{a}
\end{bmatrix}$$
\subsection*{(e)}
We have that 
$$R_11=\sqrt{a}$$
thus we must have that $a\geq 0$, and thus we also have that 
$$R_{1,2:3}=[\frac{1}{\sqrt{a}},0]$$
Therefore we have that we want to find the cholesky factorization of
of 
$$\begin{bmatrix}
    -a & 1\\
    1 & a
\end{bmatrix}-\begin{bmatrix}
    \frac{1}{a} & 0\\
    0 & 0
\end{bmatrix}=\begin{bmatrix}
    -a-\frac{1}{a} & 1\\
    1 & a
\end{bmatrix}$$
This cannot be factorized since we already have 
that $a\geq 0$ and thus we have that $-a-\frac{1}{a}\leq 0$. Thus we have
that A is not positive definite.
\subsection*{(h)}
We have that 
$$R_11=1$$
And thus
$$R_{1,2:3}=\frac{1}{1}A_{1,2:3}=[1,1]$$
Thus we have that we want to find the cholesky factorization of
$$\begin{bmatrix}
    a & a\\
    a & 2
\end{bmatrix}-\begin{bmatrix}
    1 & 1\\
    1 & 1
\end{bmatrix}=\begin{bmatrix}
    a-1 & a-1\\
    a-1 & 1
\end{bmatrix}$$
Thus we have that
$$R_{22}=\sqrt{a-1}$$
And thus we have that
$$R_{23}=\frac{a-1}{\sqrt{a-1}}=\sqrt{a-1}$$
Thus in order for these to exist we must have that $a\geq 1$, and thus we have
that 
$$R_{33}=\sqrt{1-a-1}=\sqrt{-a}$$
This cannot exist, since we have already shown that $a\geq 1$, and thus we have
that A is not positive definite and that the cholesky factorization does not
exist.
\section*{A11.14}
\subsection*{(a)}
We have that the cholensky factorization of $B$
$$B=R_B^T R_B$$
is of the form of  
$$R_B=\begin{bmatrix}
    R & v\\
    0 & v_{n+1}
\end{bmatrix}$$
Now we need to solve for $v$ and $v_{n+1}$, we have that
\begin{align*}
R_B^t R_B&=\begin{bmatrix}
    R^T & 0\\
    v^T & v_{n+1}
\end{bmatrix}\begin{bmatrix}
    R & v\\
    0 & v_{n+1}
\end{bmatrix}\\
&=\begin{bmatrix}
    R^T R & R^Tv\\
    v^TR & v^Tv+v_{n+1}^2
\end{bmatrix}
\end{align*}
Thus we have that 
$$v=R^{-T}u$$
and 
$$v_{n+1}^2=u^TA^{-1}u-1$$
and thus we have that
$$\boxed{
    R_B=\begin{bmatrix}
        R & R^{-T}u\\
        0 & \sqrt{u^TA^{-1}u-1}
    \end{bmatrix}
}$$
\subsection*{(b)}
To solve for $v$, since $R$ is a upper triangular matrix, we can just use
forward substitution to solve for $v$, which will cost us $n^2$ flops, 
then to solve for $v_{n+1}$ we take the dot product of $v$ with itself,
which will cost us $2n-1$ flops, and then subtract one and take the square 
root, which will cost us $2$ flops, thus we have that the total cost is
$\boxed{n^2+2n+1}$ flops.



\end{document}


