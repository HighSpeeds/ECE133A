\documentclass[12pt]{article}
\author{Lawrence Liu}
\usepackage{subcaption}
\usepackage{graphicx}
\usepackage{amsmath}
\usepackage{pdfpages}
\newcommand{\Laplace}{\mathscr{L}}
\setlength{\parskip}{\baselineskip}%
\setlength{\parindent}{0pt}%
\usepackage{xcolor}
\usepackage{listings}
%\definecolor{backcolour}{rgb}{0.95,0.95,0.92}
\usepackage{amssymb}
\usepackage[T1]{fontenc}
\usepackage{beramono}%\lstdefinestyle{mystyle}{
%    backgroundcolor=\color{backcolour}}
%\lstset{style=mystyle}
%\usepackage[usenames,dvipsnames]{xcolor}%%
%% Julia definition (c) 2014 Jubobs
%%



\title{ECE 133A HW 1}
\begin{document}
\maketitle
\section*{Exercise T8.8}
For a specific $t_i$ we have 
\begin{align*}
    f(t_i)&=y_i\\
    \frac{c_1+c_2t_i+c_3t_i^2}{1+d_1t_i+d-2t_i^2}&=y_i\\
    c_1+c_2t_i+c_3t_i^2=y_i(1+d_1t_i+d_2t_i^2)\\
    c_1+c_2t_i+c_3t_i^2-y_{i}d_1t_i-y_{i}d_2t_i^2=y_i
\end{align*}
Therefore we can construct a matrix $A$ and a vector $b$, such that 
$A\theta=b$, where $\theta=[c_1,c_2,c_3,d_1,d_2]^T$. We have that for 
5 values of $t_i$ and the corresponding 5 values of $y_i$,
$$A=\begin{bmatrix}
    1&t_1&t_1^2&-y_{1}t_1&-y_{1}t_1^2\\
    1&t_2&t_2^2&-y_{2}t_2&-y_{2}t_2^2\\
    1&t_3&t_3^2&-y_{3}t_3&-y_{3}t_3^2\\
    1&t_4&t_4^2&-y_{4}t_4&-y_{4}t_4^2\\
    1&t_5&t_5^2&-y_{5}t_5&-y_{5}t_5^2
\end{bmatrix}$$
and
$$B=\begin{bmatrix}
    y_1\\
    y_2\\
    y_3\\
    y_4\\
    y_5
\end{bmatrix}$$
Therefore we can then just solve for $\theta$ with $\theta=A^{-1}b$.
\\\\
Thus for the values of $t$ and $y$ given, we can get the values of $theta$
with the following code:
\lstinputlisting[
    basicstyle=\tiny, %or \small or \footnotesize etc.
]{Problem1.jl}
We get that
$$\theta=\boxed{(-6.117, 6.99, -1.322, -0.709, 0.158)}$$
\section*{Exercise T8.10}
Squaring these, we get that for $1\leq i\leq 4$:
\begin{align*}
	\rho_i^2&=||x-a_i||^2\\
	&=(x_1-a_{i1})^2+(x_2-a_{i2})^2+(x_3-a_{i3})^2
\end{align*}
Subtracting $\rho_4^2$ from the others, we get that for $1\leq i \leq 3$ 
\begin{align*}
	\rho_i^2-\rho_4^2=&
	(2a_{41}-2a_{i1})x_1+a_{i1}^2-a_{41}^2\\
	&+(2a_{42}-2a_{i2})x_2+a_{i2}^2-a_{42}^2\\
	&+(2a_{43}-2a_{i3})x_3+a_{i3}^2-a_{43}^2
\end{align*}
Therefore for the values of $\rho_i$ and $a_i$ we get that:
$$x=(0.605, 0.405, -0.503)$$
Which can be found with the following code in Julia
\lstinputlisting[
    basicstyle=\tiny, %or \small or \footnotesize etc.
]{Problem2.jl}
\section*{Exercise A3.8}
\subsection*{(a)}
We have that $f(s_k,t_k)=\sum_{i=1}^{3}\sum_{j=1}^{3}c_ijs_k^{i-1}t_k^{j-1}=y_k$, then we will have
$$A=\begin{bmatrix}
    1&t_1&t_1^2&s_1&s_1t_1&s_1t_1^2&s_1^2&s_1^2t_1&s_1^2t_1^2\\
    \vdots & \vdots & \vdots & \vdots & \vdots & \vdots & \vdots & \vdots & \vdots\\
    1&t_9&t_9^2&s_9&s_9t_9&s_9t_9^2&s_9^2&s_9^2t_9&s_9^2t_9^2
\end{bmatrix}$$
$$x=[c_11,c_12,c_13,c_21,c_22,c_23,c_31,c_32,c_33]^T$$
$$b=\begin{bmatrix}
    y_1\\
    \vdots\\
    y_9
\end{bmatrix}$$
\subsection*{(b)}
Using the code below:
\lstinputlisting[
    basicstyle=\tiny, %or \small or \footnotesize etc.
]{Problem3.jl}
We get that 
$$\begin{matrix}
    c_{11}&=4.0&c_{12}&=-8.5&c_{13}&=4.5\\
    c_{21}&=8.0&c_{22}&=-3.25&c_{23}&=1.75\\
    c_{31}&=-5.0&c_{32}&=4.75&c_{33}&=-2.25
\end{matrix}$$

\subsection*{Exercise A4.12}
\subsection*{(a)}
When $a=1$, because this would make the columns not linearly independent.
\subsection*{(b)}
We will have that 
$$A^{-1}=\begin{bmatrix}
    c_1 & 0 & \cdots & 0 & c_2\\
    0 & c_1 & \cdots & 0 & c_2\\
    \vdots & \vdots & \ddots & \vdots & \vdots\\
    0 & c_2 & \cdots & c_1 & 0\\
    c_2 & 0 & \cdots & 0 & c_1
\end{bmatrix}$$
Therefore we must have that 
$$c_1+\alpha c_2=1$$
$$c_2+\alpha c_1=0$$
thus we get that $c_2=-\alpha c_1$ and thus
$$c_1=\frac{1}{1-\alpha^2}$$
$$c_2=-\frac{\alpha}{1-\alpha^2}$$
\end{document}
