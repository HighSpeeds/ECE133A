\documentclass[12pt]{article}
\author{Lawrence Liu}
\usepackage{subcaption}
\usepackage{graphicx}
\usepackage{amsmath}
\usepackage{pdfpages}
\newcommand{\Laplace}{\mathscr{L}}
\setlength{\parskip}{\baselineskip}%
\setlength{\parindent}{0pt}%
\usepackage{xcolor}
\usepackage{listings}
%\definecolor{backcolour}{rgb}{0.95,0.95,0.92}
\usepackage{amssymb}
\usepackage[T1]{fontenc}
\usepackage{beramono}%\lstdefinestyle{mystyle}{
%    backgroundcolor=\color{backcolour}}
%\lstset{style=mystyle}
%\usepackage[usenames,dvipsnames]{xcolor}%%
%% Julia definition (c) 2014 Jubobs
%%



\title{ECE 133A HW 1}
\begin{document}
\maketitle
\section*{Exercise T2.4}
$\phi$ could be linear, from the three points, we can create a plane (a linear function) that spans all three points. However I could also create a concave or convex function that would also pass through all three points.
\section*{Exercise T2.8}
\subsection*{(a)}
\begin{align*}
\int p(x)dx&=\sum_{i=1}^{n}c_i\frac{1}{i}x^{i}\\
\int_{\alpha}^\beta&=\sum_{i=1}^{n}c_i\frac{1}{i}(\beta^i-\alpha^i)
\end{align*}
therefore we have
$$a=\boxed{\left(c_1(\beta-\alpha)),...,\frac{c_n}{n}(\beta^n-\alpha^n)\right)}$$
\subsection*{(b)}
we have
$$p'(\alpha)=\sum_{i=1}^{n}(i-1)c_i \alpha^{i-2}$$
thus we get
$$b=\boxed{\left(0,c_2,...,(n-1)\alpha^{n-2}\right)}$$
\section*{Exercise A1.2}
let $u=(\sqrt{x_1},...,\sqrt{x_n})$ and $v=\left(\sqrt{\frac{1}{x_1}},...,\sqrt{\frac{1}{x_n}}\right)$, from Cauchy-Schwarz we have
$$<u,v>^2\leq<u,u><v,v>$$
$$n^2\leq\left(\sum_{k=1}^{n}x_n\right)\left(\sum_{k=1}^{n}\frac{1}{x_k}\right)$$
$$\frac{1}{n}\sum_{k=1}^{n}x_k=n\left(\sum_{k=1}^n\frac{1}{x_k}\right)^{-1}$$
$$\frac{1}{n}\sum_{k=1}^{n}x_k=\left(\frac{1}{n}\sum_{k=1}^n\frac{1}{x_k}\right)^{-1}$$
\section*{Exercise T3.25}
\subsection*{(a)}
\begin{align*}
	E[p]&=\boxed{\theta\mu+(1-\theta)\mu^{\text{rf}}\textbf{1}}
\end{align*}
\begin{align*}
	Var(p)&=\theta^2\sigma^2\\
	\sqrt{Var(p)}&=\boxed{|\theta|\sigma}
\end{align*}
\end{document}
