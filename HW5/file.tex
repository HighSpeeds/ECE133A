\documentclass[12pt]{article}
\author{Lawrence Liu}
\usepackage{subcaption}
\usepackage{graphicx}
\usepackage{amsmath}
\usepackage{pdfpages}
\newcommand{\Laplace}{\mathscr{L}}
\setlength{\parskip}{\baselineskip}%
\setlength{\parindent}{0pt}%
\usepackage{xcolor}
\usepackage{listings}
%\definecolor{backcolour}{rgb}{0.95,0.95,0.92}
\usepackage{amssymb}
\usepackage[T1]{fontenc}
\usepackage{beramono}%\lstdefinestyle{mystyle}{
%    backgroundcolor=\color{backcolour}}
%\lstset{style=mystyle}
%\usepackage[usenames,dvipsnames]{xcolor}%%
%% Julia definition (c) 2014 Jubobs
%%



\title{ECE 133A HW 4}
\begin{document}
\maketitle
\section*{Exercise T13.3}
\subsection*{(a)}
With the following julia code we get:
\lstinputlisting[
    basicstyle=\tiny, %or \small or \footnotesize etc.
]{problem1.jl}
that 
$$\theta_1=3.125592633829346$$
and 
$$\theta_2=0.1540181798438225$$
which results in the following fit:\\
\includegraphics[scale=0.5]{"Moore's Law.png"}
\subsection*{(b)}
From our fit we expect the number of transistors to be:
$$10^{\theta_1+\theta_2(2015-1970)}\approx 10^{10}$$
Which is more than the acutally number of $4\cdot 10^9$ transistors:
\subsection*{(c)}
This is in line with Moore's law since
$2\theta_2=0.30803635968$ which is close to $\log_{10}(2)=0.30102999566$
\section*{Exercise T12.12}
\subsection*{(a)}
\section*{Exercise A8.3}
We can get that 
$$\alpha t_i+\beta=\ln(\frac{y_i}{1-y_i})$$
So therefore we can have a least squares problem, with
$$A=\begin{bmatrix}
t_1 & 1\\
t_2 & 1\\
\vdots & \vdots\\
t_n & 1
\end{bmatrix}$$
and 
$$b=[\ln(\frac{y_1}{1-y_1}),\ln(\frac{y_2}{1-y_2}),\dots,\ln(\frac{y_n}{1-y_n})]^T$$
and
$$x=[\alpha,\beta]^T$$
Then we have a least squares problem of 
$$||Ax-b||^2$$

\section*{Exercise A8.12}
\subsection*{(a)}
$$f(y)=||Ay-b||^2+(c^Ty-d)^2$$
To minimize we take the derivative of it with respect to $y_i$ for all
$1\leq n \leq N$ and set it to zero have
$$\frac{\partial}{\partial y_i} f(y)=2(A^T(Ay-b))_i+2(c^Ty-d)c_i=0$$
Thus we have 
$$\nabla f(y)=2(A^T(Ay-b)+c(c^Ty-d))=0$$
which gives us
\begin{align*}
    \nabla f(y)&=0\\
    2(A^T(Ay-b)+c(c^Ty-d))&=0\\
    A^T(Ay-b)+c(c^Ty-d)&=0\\
\end{align*}
if $\hat{y}$ is a solution then we must have that
$$A^T(A\hat{y}-b)+c(c^T\hat{y}-d)=0$$
we can confirm this, since 
$$\hat{y}=\hat{x}+\frac{d-c^T\hat{x}}{1+c^T(A^TA)^{-1}c}(A^TA)^{-1}c$$
we have:
\begin{align*}
    A^T(A\hat{y}-b)+c(c^T\hat{y}-d)&=0\\
    A^TA\frac{d-c^T\hat{x}}{1+c^T(A^TA)^{-1}c}(A^TA)^{-1}c+cc^T\hat{x}+c(c^T\frac{d-c^T\hat{x}}{1+c^T(A^TA)^{-1}c}(A^TA)^{-1}c-d)&=0\\
    dc-c^T\hat{x}c+cc^T(d-c^T\hat{x})(A^TA)^{-1}c-c(d-c^T\hat{x})(1+c^T(A^TA)^{-1}c)&=0\\
    cc^T(d-c^T\hat{x})(A^TA)^{-1}c-c(d-c^T\hat{x})(c^T(A^TA)^{-1}c)&=0\\
    cc^T(d-c^T\hat{x})(A^TA)^{-1}c-cc^T(d-c^T\hat{x})(A^TA)^{-1}c&=0
\end{align*}
\subsection*{(b)}
We first compute the QR factorization fo $A$, which will cost us $2mn^2$ flops, 
then we can compute 
$\hat{x}$ with an additonal $2mn+n^2$ flops. Likewise,
since we can rewrite $(A^TA)^{-1}c$ as $(R^TQ^TQR)^{-1}c=(R^TR)^{-1}c$, which 
we can solve in $2n^2$ flops. then computing $c^T\hat{x}$ and $c^T(A^TA)^{-1}c$ will
each cost us an additional $2n-1$ flops, then computing $\frac{d-c^T\hat{x}}{1+c^T(A^TA)^{-1}c}$ will 
cost us 3 flops.Then computing $\hat{x}+\frac{d-c^T\hat{x}}{1+c^T(A^TA)^{-1}c}(A^TA)^{-1}c$
will cost us $2n$ flops, so in total this algorithm will cost us 
$\boxed{2mn^2+2mn+3n^2+6n-1}$ flops.




\end{document}
