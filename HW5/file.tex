\documentclass[12pt]{article}
\author{Lawrence Liu}
\usepackage{subcaption}
\usepackage{graphicx}
\usepackage{amsmath}
\usepackage{pdfpages}
\newcommand{\Laplace}{\mathscr{L}}
\setlength{\parskip}{\baselineskip}%
\setlength{\parindent}{0pt}%
\usepackage{xcolor}
\usepackage{listings}
%\definecolor{backcolour}{rgb}{0.95,0.95,0.92}
\usepackage{amssymb}
\usepackage[T1]{fontenc}
\usepackage{beramono}%\lstdefinestyle{mystyle}{
%    backgroundcolor=\color{backcolour}}
%\lstset{style=mystyle}
%\usepackage[usenames,dvipsnames]{xcolor}%%
%% Julia definition (c) 2014 Jubobs
%%



\title{ECE 133A HW 4}
\begin{document}
\maketitle
\section*{Exercise T13.3}
\subsection*{(a)}
With the following julia code we get:
\lstinputlisting[
    basicstyle=\tiny, %or \small or \footnotesize etc.
]{problem1.jl}
that 
$$\theta_1=3.125592633829346$$
and 
$$\theta_2=0.1540181798438225$$
which results in the following fit:\\
\includegraphics[scale=0.5]{"Moore's Law.png"}
\subsection*{(b)}
From our fit we expect the number of transistors to be:
$$10^{\theta_1+\theta_2(2015-1970)}\approx 10^{10}$$
Which is more than the acutally number of $4\cdot 10^9$ transistors:
\subsection*{(c)}
This is in line with Moore's law since
$2\theta_2=0.30803635968$ which is close to $\log_{10}(2)=0.30102999566$
\section*{Exercise T12.2}
\end{document}
